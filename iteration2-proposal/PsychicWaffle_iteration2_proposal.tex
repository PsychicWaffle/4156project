\documentclass{article} 
\usepackage{url, graphicx}
\usepackage[margin=1in]{geometry}
\usepackage{textcomp}
\usepackage{algpseudocode}
\usepackage{algorithm}
\usepackage{titling}
\usepackage{amsmath}
\usepackage{amssymb}
\usepackage{amsthm}
\usepackage{verbatim}
\usepackage{listings} % for code highlighting/formatting
\usepackage[final]{pdfpages}

\usepackage{color} %defining colors for syntax highlighting
\definecolor{syntaxBlue}{RGB}{42,0.0,255}
\definecolor{syntaxGreen}{RGB}{63,127,95}
\definecolor{syntaxPurple}{RGB}{127,0,85}
\definecolor{syntaxCyan}{RGB}{0,155,155}
\definecolor{syntaxGreyBg}{RGB}{220,220,220}


\lstset{
    basicstyle=\small\ttfamily, % Global Code Style
    tabsize=2, % number of spaces indented when discovering a tab 
    columns=fixed, % make all characters equal width
    keepspaces=true, % does not ignore spaces to fit width, convert tabs to spaces
    showstringspaces=false, % lets spaces in strings appear as real spaces
    breaklines=true, % wrap lines if they don't fit
    frame=trbl, % draw a frame at the top, right, left and bottom of the listing
    frameround=tttt, % make the frame round at all four corners
    framesep=4pt, % quarter circle size of the round corners
    numbers=left, % show line numbers at the left
    numberstyle=\tiny\ttfamily, % style of the line numbers
    commentstyle=\color{syntaxGreen},
    keywordstyle=\color{syntaxPurple},
    stringstyle=\color{syntaxBlue},
    emph={int,char,float,struct,string},
    emphstyle=\color{syntaxCyan},
    backgroundcolor=\color{syntaxGreyBg},
}

\title{Psychic Waffle Project Proposal}
\author{
    A.J. Feather\\
    \texttt{af2849@columbia.edu}
    \and
    Andrew Grant\\
    \texttt{amg2215@columbia.edu}
    \and
    Jacob Graff\\
    \texttt{jag2302@columbia.edu}
    \and
    Jake Weissman\\
    \texttt{jdw2159@columbia.edu}
}
\date{\today}

\begin{document}

\maketitle

\section{Iteration 2 Overview}

For our first iteration, we focused on getting a working brute force solution for the trade execution algorithm up and running to demo to our customer. This went well, and we were able to produce a demo that our customer was happy with. We have a working web application that allows users to execute trades, monitor progress, and view transaction history.

Our demo with the customer went really well and we got some good feedback to help guide our second iteration. It was clear from our customer that the next priority he was interested in was a more developed algorithm to execute the entire transaction over the course of one market day. It was also indicated that some improvements in the UI and transaction history would be helpful. Per this feedback, much of our second iteration will be focused on an improved algorithm. We will also work on improving UI views but this will take a secondary priority.

\section{User Stories / Trello Board}

Our trello board that contains all of our second iteration user stories can be found here: 

https://trello.com/b/ZAo59Z8n/2nd-iteration-j-p-morgan-project

\section{Wireframes}

\section{Class Diagram}
Since the improved algorithm does not require any new classes, much of our class diagram remains unchanged from the first iteration. It can be found here with some minor modifications:

https://github.com/andyg7/4156Project/blob/master/docs/UML-CLASS-DIAGRAM.pdf

\end{document} 