\documentclass{article} 
\usepackage{url, graphicx}
\usepackage[margin=1in]{geometry}
\usepackage{textcomp}
\usepackage{algpseudocode}
\usepackage{algorithm}
\usepackage{titling}
\usepackage{amsmath}
\usepackage{amssymb}
\usepackage{amsthm}
\usepackage{verbatim}
\usepackage{hyperref}
\usepackage{listings} % for code highlighting/formatting
\usepackage[final]{pdfpages}

\usepackage{color} %defining colors for syntax highlighting
\definecolor{syntaxBlue}{RGB}{42,0.0,255}
\definecolor{syntaxGreen}{RGB}{63,127,95}
\definecolor{syntaxPurple}{RGB}{127,0,85}
\definecolor{syntaxCyan}{RGB}{0,155,155}
\definecolor{syntaxGreyBg}{RGB}{220,220,220}


\lstset{
    basicstyle=\small\ttfamily, % Global Code Style
    tabsize=2, % number of spaces indented when discovering a tab 
    columns=fixed, % make all characters equal width
    keepspaces=true, % does not ignore spaces to fit width, convert tabs to spaces
    showstringspaces=false, % lets spaces in strings appear as real spaces
    breaklines=true, % wrap lines if they don't fit
    frame=trbl, % draw a frame at the top, right, left and bottom of the listing
    frameround=tttt, % make the frame round at all four corners
    framesep=4pt, % quarter circle size of the round corners
    numbers=left, % show line numbers at the left
    numberstyle=\tiny\ttfamily, % style of the line numbers
    commentstyle=\color{syntaxGreen},
    keywordstyle=\color{syntaxPurple},
    stringstyle=\color{syntaxBlue},
    emph={int,char,float,struct,string},
    emphstyle=\color{syntaxCyan},
    backgroundcolor=\color{syntaxGreyBg},
}

\title{Demo}
\author{
    A.J. Feather\\
    \texttt{af2849@columbia.edu}
    \and
    Andrew Grant\\
    \texttt{amg2215@columbia.edu}
    \and
    Jacob Graff\\
    \texttt{jag2302@columbia.edu}
    \and
    Jake Weissman\\
    \texttt{jdw2159@columbia.edu}
}
\date{\today}

\begin{document}

\maketitle

\section{Demo Plan}
\subsection{Intro}
In our demo, we will first demonstrate our UI, by showing the various capabilities and data displays. This includes such features as multiple users and user tracking. Next, we will explain how the algorithm works, by submitting more orders and explaining the logic behind them. If time permits we will also discuss our testing suite and continuous integration tools.
\subsection{UI}
\begin{enumerate}
\item Load localhost:5000/home
\item Enter 1000 shares in Number of Shares field
\item Click Execute
\item Select Market
\item Enter 100 shares
\item Click Execute
\item Select Limit
\item Enter 1000 shares
\item Enter min price near current market price
\item Click Execute
\item Wait for transactions to run
\item Click View History
\item Enter reasonable date ranges
\item Click Submit
\item Click Logout
\item Click Create User
\item Enter new username and password
\item Click Submit
\item Enter same username and password
\item Click Change Password
\item Enter proper data, including new password
\item Click Submit
\item Demo complete
\end{enumerate}
\subsection{Algorithm}
Demoing the Algorithm
\begin{enumerate}
\item Submit a new time-weighted order of size 100,000 by inputing order size and clicking time-weighted button
\item Submit a new time-weighted order of size 1000 by inputing order size and clicking time-weighted button
\item Watch it show up in the ``Running Transactions'' box and see sub orders get executed over time
\item Submit a new market order, and watch it get executed straight away. See it show up in the ``Todays Completed Transaction" section
\item Submit a new limit order with a min price, above the current market price, and watch it wait to execute any orders until price rises
\item Observe orders getting executed and how many shares are getting sold over time
\item Observe how orders of different order sizes get executed. For example, see how the order with size 100,000 is selling more units more frequently than the order with order size 100
\end{enumerate}
Explaining the Algorithm
\begin{itemize}
\item Three types of orders: time-weighted, limit, market
\item Market order = sell total amount right now
\item Time-weighted = break order up into small chunks and sell throughout the day
\begin{itemize}
\item The algorithm will first determine how long the intervals between sub orders needs to be in order to sell 1 unit at a time. If this interval is larger than 5 minutes, then the algorithm will simply sell 1 unit every x minutes. e.g. if there is a sell order of 10 units over the course of the day, it will sell 1 unit every 1 hour (approximately)
\item If the calculated time interval is too small (e.g. if the order is big and would need to sell 1 unit every 30 seconds), it will figure out how much it needs to sell for it to sell every 5 minutes. For example, for a huge order it may decide it needs to sell 500 units every 5 minutes. The algorithm will then 
\end{itemize}
\item limit order is like time-weighted order, except there is a min price. Whenever a worker fails to execute an order because the price is too low, it will recalculate order size for the next order  
\item Explain how orders of previous day are cancelled
\item Explain that orders dynamically recalculate when to sell next and the next order size, so that orders can handle the system being paused, stopped etc. That is, the algorithm ultimately relies on two inputs for executing a transaction: 1) time left in day 2) number of shares left to sell.  
\end{itemize}


\end{document} 