\documentclass{article} 
\usepackage{url, graphicx}
\usepackage[margin=1in]{geometry}
\usepackage{textcomp}
\usepackage{algpseudocode}
\usepackage{algorithm}
\usepackage{titling}
\usepackage{amsmath}
\usepackage{amssymb}
\usepackage{amsthm}
\usepackage{verbatim}
\usepackage{hyperref}
\usepackage{listings} % for code highlighting/formatting
\usepackage[final]{pdfpages}

\usepackage{color} %defining colors for syntax highlighting
\definecolor{syntaxBlue}{RGB}{42,0.0,255}
\definecolor{syntaxGreen}{RGB}{63,127,95}
\definecolor{syntaxPurple}{RGB}{127,0,85}
\definecolor{syntaxCyan}{RGB}{0,155,155}
\definecolor{syntaxGreyBg}{RGB}{220,220,220}


\lstset{
    basicstyle=\small\ttfamily, % Global Code Style
    tabsize=2, % number of spaces indented when discovering a tab 
    columns=fixed, % make all characters equal width
    keepspaces=true, % does not ignore spaces to fit width, convert tabs to spaces
    showstringspaces=false, % lets spaces in strings appear as real spaces
    breaklines=true, % wrap lines if they don't fit
    frame=trbl, % draw a frame at the top, right, left and bottom of the listing
    frameround=tttt, % make the frame round at all four corners
    framesep=4pt, % quarter circle size of the round corners
    numbers=left, % show line numbers at the left
    numberstyle=\tiny\ttfamily, % style of the line numbers
    commentstyle=\color{syntaxGreen},
    keywordstyle=\color{syntaxPurple},
    stringstyle=\color{syntaxBlue},
    emph={int,char,float,struct,string},
    emphstyle=\color{syntaxCyan},
    backgroundcolor=\color{syntaxGreyBg},
}

\title{Psychic Waffle Project Proposal}
\author{
    A.J. Feather\\
    \texttt{af2849@columbia.edu}
    \and
    Andrew Grant\\
    \texttt{amg2215@columbia.edu}
    \and
    Jacob Graff\\
    \texttt{jag2302@columbia.edu}
    \and
    Jake Weissman\\
    \texttt{jdw2159@columbia.edu}
}
\date{\today}

\begin{document}
\[\documentclass{article}
\usepackage[utf8]{inputenc}
\usepackage{hyperref}
\hypersetup{
    colorlinks=true,
    linkcolor=blue,
    filecolor=magenta,      
    urlcolor=cyan,
}
\begin{document}
\section{Continuous Integration}
We used Travis CI as our integration tool and linked it to our GitHub repo. We set it up so that on each commit and push, Travis runs our entire test suite and reports on the coverage statistics as well.

\section{Code Coverage}
Since we have been gradually writing tests for our app throughout the development process, when we integrated the Travis CI tool and started more closely monitoring our code coverage, we were looking at around 50\% statement coverage. This was a good place to start but nowhere near where we wanted to be. We set out on a quest to get as close to 100\% coverage as possible. We mocked out a lot of system components and were able to simulate interactions in the system, including database and network communication, that made it possible to test much more thoroughly. Most of our python files are showing at least 75\% coverage and some are showing close to 100\% as some files were easier to test than others. Our most recent coverage report can be found using the link below. Unfortunately, we did not start tracking this document until we were almost done improving it so the history does not show how far we came over the last few days. Anyway, we were ultimately able to get our coverage up to 82\% statement coverage! This was a large undertaking and we are very proud with how far our testing suite came.

<<<<<<< HEAD
\subsection{Report:}

\begin{verbatim}

$ coverage report -m
Name                                Stmts   Miss  Cover   Missing
-----------------------------------------------------------------
app/__init__.py                         0      0   100%
app/database_methods.py               167      2    99%   184-186
app/database_objects.py                27      0   100%
app/market_methods.py                  65     15    77%   16-18, 26-28,
39-41, 54-56, 82-84
app/multi_processing_handler.py        20     14    30%   7-8, 12-17, 21-27
app/order.py                          148     15    90%   38, 69-71, 107,
112-113, 116-117, 133, 145, 189-190, 207, 210
app/server.py                         206     54    74%   37-56, 71-72, 101-125,
132, 163, 182-184, 195-205, 279,288, 290-291, 294-295, 302, 320-331
app/transaction.py                    109     30    72%   60-61, 67-69, 71-80,
114-120, 123-140, 160-161, 181
app/validity_checker.py                39      4    90%   12-13, 29, 31
-----------------------------------------------------------------
TOTAL                                 781    134    83%

\end{verbatim}

\subsection{Remaining Problems}


We had trouble running full system tests due to what was required in order to properly ping the server programmatically. This was also an issue with the multiprocessing code because we needed to ping the server in order to get it to run properly. However, there were a number of system tests included in the unit test files where we created, deleted and waiting for transactions to properly complete. In addition, we tested all of the internal classes and functions as best we could and our coverage takes care of all code we could test successfully via unit tests.

To make sure we test the full server and remaining code we performed the following system tests. Happily, there were no errors while performing the below full system tests, which indicates our unit tests and existing system tests are robust.

\subsection{Performed System Tests}

User story: As a user I want to be able to create a user profile and login/logout securely

Test: included in test folder

User story: As a user I want to be able to let the system complete a sell order automatically

Test: included in the test folder

User Story (2): As a user I want to be able to view the currently executing transactions, and As a user I want to be able to see past completed history

\begin{enumerate}
\item 
Test:
\item login using known username
\item amount to sell
\item select price
\item select limit
\item submit order
\item repeat steps 2 through 5 using a price below current market price, above current market price and at current market price
\item view history and verify transactions executed properly (large limit should not have sold)
\item repeat steps 2-6 excluding 3 and selecting market instead of limit in 4
\item repeat steps 2-6 excluding 3 and selecting time-weighted instead of limit in 4 
\item view history and verify transactions executed properly
\item logout
\item login
\item view history and verify it still contains correct values
\end{enumerate}

\section{Resources}
\begin{itemize}
	\item Task board = \url{https://trello.com/b/ZAo59Z8n/2nd-iteration-j-p-morgan-project}
	\item Repo = \url{https://github.com/PsychicWaffle/4156project}
=======
While we made a ton of progress, we still fell short of our goal of reaching 100\% coverage. With a lot more time we might have been able to get it there but we were starting to see diminishing returns at a certain point. Some issues that prevented us from reaching our goal was some areas of the code with hard-coded values. For example, we hard-coded the URL to the market exchange which did not allow us to test for cases when the exchange was down. This would have been possible with a more configurable environment, but we chose not to make those changes at the risk of messing up the current working state of the code right before time to demo. This issue played itself out in a few more cases where it was hard to force certain error conditions. Additionally, the code that handles the multiprocessing queue system was hard to test and hurt our coverage numbers.

\section{System Tests}

\section{Resources}
\begin{itemize}
\item Code Coverage Report = \url{https://github.com/PsychicWaffle/4156project/blob/master/docs/coverage_report.txt}
\item Test Suite = \url{https://github.com/PsychicWaffle/4156project/tree/master/code/tests}
\item Task board = \url{https://trello.com/b/ZAo59Z8n/2nd-iteration-j-p-morgan-project}
\item Repo = \url{https://github.com/PsychicWaffle/4156project}
\item Issues Tracker = \url{https://github.com/PsychicWaffle/4156project/issues}
\item Demo Docs = \url{https://github.com/PsychicWaffle/4156project/tree/master/docs/demo-docs}
>>>>>>> aa0296f95636be82a9a9ac97ca5ab35935005d8d
\end{itemize}

\end{document}


\]
\end{document}